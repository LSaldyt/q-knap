\documentclass{article}

\usepackage{indentfirst}
\usepackage{setspace}
\doublespacing

% ==================================================
% Paper
% ==================================================

\title{Senior Experience Reflective Essay}
\date{05-11-2017}
\author{Lucas Saldyt}

\begin{document}

\maketitle
\pagenumbering{gobble}
\newpage
\pagenumbering{arabic}

% ==================================================
\section{Reflections}
% ==================================================

During this project, I've read more than 30 publications: From Bell Labs, Sandia Labs, Los Alamos Labs, Harvard, Carnegie Mellon, MIT, IBM, NASA, and Microsoft.
For every problem I've solved, I've probably found another dozen that are worth pursuing.
Quantum Computing is a vast field, and I'm lucky to have been exposed to even a small percentage of it.

As this project comes to a close, I feel that I've gained a layprogrammers perspective of running NP-Complete problems on different types of Quantum Computers.
Since I'm unaffiliated with either IBM, Los Alamos, or DWave, I've been able to approach the problem from a neutral perspective, and recognize both the flaws and benefits of each implementation.
In addition, I've grown close to the problem itself. I see instances of it in my daily life, and am continually impressed by how subtle it is.
For instance, while I was completing my senior experience I was simultaneously hunting craigslist for used cars.
I wanted to find a car with good mileage, high safety, enjoyable peripherals (like a stereo), and a low odometer reading, for the lowest price possible. 

Surprisingly, this is essentially an instance of the knapsack problem! 
At home, at school, and in business, the Knapsack problem shows up.
I appreciate the importance of solving this problem using Quantum Computation, and will be continuing my research with Los Alamos if possible, as well as working on similar problems at Sandia.

\end{document}
