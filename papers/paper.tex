\documentclass{article}

\usepackage{indentfirst}
\usepackage{setspace}
\doublespacing

% ================================================================================= 
% Package for showing source code
% ================================================================================= 

\usepackage{listings}
\usepackage{color}

\definecolor{dkgreen}{rgb}{0,0.6,0}
\definecolor{gray}{rgb}{0.5,0.5,0.5}
\definecolor{mauve}{rgb}{0.58,0,0.82}

\lstset{frame=tb,
language=C,
aboveskip=.5mm,
belowskip=.5mm,
showstringspaces=false,
columns=flexible,
basicstyle={\scriptsize\ttfamily},
numbers=none,
numberstyle=\tiny\color{gray},
keywordstyle=\color{blue},
commentstyle=\color{dkgreen},
stringstyle=\color{mauve},
breaklines=false,
breakatwhitespace=true,
tabsize=3
}

% ================================================================================= 
% Package for flowcharts/diagrams
% ================================================================================= 

\usepackage{tikz}
\usetikzlibrary{shapes.geometric, arrows}

\tikzstyle{startstop} = [rectangle, rounded corners, minimum width=3cm, minimum height=1cm,text centered, draw=black, fill=red!30]
\tikzstyle{io}        = [rectangle, minimum width=3cm, minimum height=1cm,text centered, draw=black, fill=blue!30]
\tikzstyle{process}   = [diamond, minimum width=2cm, minimum height=0cm, text centered, draw=black, fill=orange!30]
\tikzstyle{arrow}     = [thick,->,>=stealth]

% ==================================================
% Paper
% ==================================================

\title{DWave Interface for NP-complete problems}
\date{05-10-2017}
\author{Lucas Saldyt}

\begin{document}

\maketitle
\pagenumbering{gobble}
\newpage
\pagenumbering{arabic}

% ==================================================
\section{Abstract}
% ==================================================

Programming a quantum annealer to solve certain kinds of NP-complete problems is actually much simpler than one would expect.
Because of the rich software toolchains surrounding the DWave, it is easy to describe problems for the annealer to solve.
This paper compares various implementations of the Knapsack problem (and lightly covers other NP-complete problems).

% ==================================================
\section{Introduction}
% ==================================================

...

% ==================================================
\section{Methods}
% ==================================================

\begin{verbatim}
Classical solving:
    Python:
        Naive
        Dynamic Programming
        Polynomial Approximation

Quantum/Simulated solving:
    Verilog
    Prolog
    Grover's (qasm)

csv problem instance input(s)
\end{verbatim}

% ==================================================
\section{Discussion}
% ==================================================

Which interface is simplest? Easiest to use?
Simple interface must be justified: If it isn't useful for solving problems, then the simplicity of the interface doesn't matter.
Even if a simpler interface works for certain kinds of problems, be careful when making claims.

Cite other benchmarks, don't do many.

% ==================================================
\section{Appendix}
% ==================================================

\begin{thebibliography}{9}
    \bibitem{gps}
        A. Newell, J. C. Shaw, H. A. Simon
        \textit{Report on a general problem-solving program}
        The Rand corporation, Santa Monica, California, 1958
\end{thebibliography}

\end{document}
